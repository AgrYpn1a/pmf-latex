\documentclass[11pt]{article}

\usepackage[serbian]{babel}

% Math
\usepackage{amsmath}
\usepackage{amssymb}
\usepackage{amsthm}

\newtheorem{definition}{Definicija}
\newtheorem{theorem}{Teorema}


\begin{document}

\begin{center}
    {\LARGE Kratak uvod u matrice i determinante \\}
    \vspace{0.6cm}
    {\Large Dragan Mašulović} \\
    \vspace{0.4cm}
    {\Large 7. maj 2020.}
\end{center}
\vspace{0.5cm}
\begin{abstract}
    Analiziranje rešenja sistema jednačina nas vodi do pojma determinante, čemo je posvećen početak
poglavlja. Nakon toga uvodimo pojam matrice i pokazujemo da je najsločenija i najneintuitivnija
operacija nad matricama, množenje matrica, zapravo posledica analize ponašanja linearnih transformacija.
Rešavanje sistema linearnih jednačina pokazujemo koristeći nekoliko strategija, od kojih neke imaju
samo teorijski značaj, i glavu zaključujemo demonstracijom tehnike za nalaženje inverzne matrice date matrice.
\end{abstract}

\section{Rešavanje sistema jednačina}
Rešavanje sistema linearnih jednačina predstavlja važnu
temu u savremenoj matematici jer se mnogi fenomeni u nauci mogu modelovati
sistemima linearnih jednačina. Tipičan primer predstavlja analiza
električnih kola primenom Kirhofovih zakona \cite{BOYD}.
Drugi važan izvor motivacije za razmatranja u ovom poglavlju su linearne transformacije \cite{AXLER}.
Recimo, Lorencova transformacija opisuje promenu parametara (položaja i vremena) materijalne tačke
u jednom referentnom sistemu kada se ona posmatra iz drugog referentnog sistema. Ako pretpostavimo
da se referentni sistem u kome se nalazi materijalna tačka kreće konstantnom brzinom $v$, pravolinijski
duž $x$-ose referentnog sistema posmatrača, onda se veza izme\dj u koordinata $(x, y, z, t)$ materijalne tačke
u njenom referentnom sistemu i koordinata $(x', y', z', t')$ u referentnom sistemu posmatrača
može opisati na sledeći način:

\begin{eqnarray*}
    x' &=& \gamma(x - vt) \\
    y' &=& y \\
    z' &=& z \\
    t' &=& \frac{\gamma}{c^2}(c^2t - vx)
\end{eqnarray*}

gde je $c$ brzina svetlosti i $\gamma = \frac{1}{\sqrt{1 - \frac{v^2}{c^2}}}$. Primetimo da su
$c$, $v$ i $\gamma$
konstante. Ovo je primer transformacije koordinata $(x, y, z, t) \mapsto (x', y', z', t')$
Lorencove transformacije samo specijalan slučaj opšte linearne tranformacije $(x_1, x_2, \ldots, x_n) \mapsto (y_1, y_2, \ldots, y_m)$:

\begin{eqnarray*}
    y_1 &=& a_{11}x_1 + a_{12}x_2 + \ldots + a_{1n}x_n \\
    y_2 &=& a_{21}x_1 + a_{22}x_2 + \ldots + a_{2n}x_n \\
    \vdots \\
    y_m &=& a_{m1}x_1 + a_{m2}x_2 + \ldots + a_{mn}x_n,
\end{eqnarray*}
gde su $a_{ij}$ neki realni brojevi.

\section{Definicija determinante}
\begin{definition}
    Neka je $n$ prirodan broj i neka su $a_{ij}$ proizvoljni realni brojevi, $1 \leq i, j \leq n$.
Determinanta reda $n$ je algebarski izraz oblika
\[
    D = \left|
    \begin{array}[]{cccc}
        a_{11} & a_{12} & \ldots & a_{1n} \\
        a_{21} & a_{22} & \ldots & a_{2n} \\
        \vdots & \vdots & \hfill & \vdots \\
        a_{n1} & a_{n2} & \ldots & a_{nn}
    \end{array}
    \right| = 
    \sum_{f \in S_n} (-1)^{\mathrm{inv}(f)} a_{1f_1}a_{2f_2} \ldots a_{nf_n},
\]
gde je $S_n$ skup svih permutacija skupa $\{1, 2, \ldots, n\}$.
Kraće pišemo i $D = |a_{ij}|_{n \times n}$.
\end{definition}
\noindent
Tri kratka komentara:

\begin{enumerate}
    \item kao što smo i nagovestili, sumiranje se vrši po svim permutacijama skupa $\{1, 2, \ldots, n\}$;

    \item zato je determinanta reda $n$ algebarski izraz koji ima $n!$ sabiraka;
    
    \item ako je $f$ parna permutacija onda je broj $\mathrm{inv}(f)$ paran, pa je $(-1)^{\mathrm{inv}(f)} = 1$;
    ako je $f$ neparna permutacija onda je broj $\mathrm{inv}(f)$ neparan, pa je $(-1)^{\mathrm{inv}(f)} = -1$;
    tako smo zapisali činjenicu da se sabirci koji odgovaraju parnim permutacijama uzimaju sa znakom $+$,
    dok se sabirci koji odgovaraju neparnim permutacijama uzimaju sa znakom $-$.
\end{enumerate}

\pagebreak
\section{Osobine determinanti}
\begin{theorem}[{\textbf{Vrsta/kolona koja se sastoji od nula}}]
Ako se u determinanti $d$ neka vrsta ili kolona sastoji isključivo od nula, onda je $D = 0$.
\end{theorem}
\begin{proof}[{\normalfont Dokaz}]
    Primetimo, prvo, da svaki sabirak u definiciji determinante sadrži tačno jedan element svake
    vrste determinante, i tačno jedan element svake kolone determinante.
    To je ključni sastojak u dokazu koji sledi.
    Neka se u determinanti $D$ vrsta $i$ sastoji isključivo od nula:
    $a_{i1} = a_{i2} = \ldots = a_{in} = 0$.
    Tada je $a_{if_i} = 0$ za svako $f \in S_n$ što znači da je svaki sabirak u izrazu
    \[
        \sum_{f \in S_n} (-1)^{\mathrm{inv}(f)} a_{1f1} \ldots a_{ifi} \ldots a_{nfn}
    \]
    jednak nuli. Dakle, $D = 0$.  
\end{proof}

Osim ove osobine, determinante imaju još niz važnih osobina koje koristimo da bismo
efektivno odrediti vrednost determinante. Sada samo navodimo osobine i tek po koji dokaz, dok ostale, često veoma komplikovane
dokaze dajemo u sledećem odeljku za posebno motivisane čitaoce.

\begin{theorem}[{\textbf{Zamena mesta dvema vrstama/kolonama determinante}}]
    Neka je $D'$ determinanta koja se dobija od determinante $D$ tako što dve vrste (ili dve kolone)
    zamene mesta. Tada je $D' = -D$.
\end{theorem}

\begin{theorem}[{\textbf{Dve jednake vrste/kolone}}]
  Ako su u determinanti $D$ neke dve vrste jednake, ili neke dve kolone jednake, onda je $D = 0$.
\end{theorem}

\section{Kramerova teorema}

Kramerova teorema predstavlja važan teorijski rezultat koji nam opisuje strukturu rešenja
kvadratnih sistema linearnih jednačina. Ova teorema nije pogodna za efikasno rešavanje sistema
linearnih jednačina i zato je njen značaj isključivo teorijski.
Posmatrajmo sledeći sistem $n$ linearnih jednačina sa $n$ nepoznatih:


\begin{thebibliography}{xx}
    \bibitem{AXLER} Sheldon Axler: ``Linear Algebra Done Right'' (2nd Ed.), Springer-Verlag New York Berlin Heidelberg 1997.
    \bibitem{BOYD} Stephen Boyd, Lieven Vandenberghe: ``Introduction to Applied Linear Algebra: Vectors, Matrices, and Least Square'',
Cambridge University Press 2018.
    \bibitem{KLEIN} Philip N. Klein: ``Coding the Matrix -- Linear Algebra through Applications to Computer Science''.
Newtonian Press 2015.
\end{thebibliography}    

\end{document}

