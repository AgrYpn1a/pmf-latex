\documentclass[12pt, a4paper]{article}


% Preamble
\usepackage[serbian]{babel}

% Math stuff
\usepackage{amsthm}
\usepackage{amssymb}

\newtheorem*{theorem*}{Teorema}
\newtheorem{theorem}{Teorema}
\newtheorem*{lemma*}{Lema}

\newcommand{\doubleint}{\mathop{\int\!\!\int}}

\begin{document}


    % % Exponents, indices, fractions...
    % $$  
    %     2^{2^n} \quad \mathrm{and} \quad x^{y^z}_{k_n}
    % $$

    % math\textsuperscript{mode}

    % $$
    %     \frac{x}{y}, \frac{2}{y}, \quad \mathrm{and} \quad \frac{x}{3}
    % $$

    % % \displaystyle; \textstyle, \scriptstyle, \scriptscriptstyle
    % \begin{displaymath}
    %     {x+ \scriptstyle\frac{1}{x} \above1pt \displaystyle y+\frac{1}{y}}
    % \end{displaymath}

    % \[
    %     {x+\frac{1}{x} \above1pt y+\frac{1}{y}}
    % \]

    % \[
    %     \sqrt{x}  
    % \]

    % \[
    %     \sqrt{\vphantom{b}a} + \sqrt{b}
    % \]

    % Functions

    % % Functions
    % \[
    %     {\displaystyle\doubleint \above0pt \scriptstyle \{ x \in X : \|x\| \leq 1 \}}
    % \]

    \section*{Zadatak 6a}


    \begin{theorem*}
        Neka su $a, b \in \mathbb{R}$ takvi da je $a \geq b$ i $b \geq 0$. Tada je
        \[
            \frac{a+b}{2} \geq \sqrt{ab}.
        \]
    \end{theorem*}

    \begin{proof}
        Primetimo, prvo, da je
        \[
            (\sqrt{a} - \sqrt{b})^2 \geq 0
        \]
        zato \v{s}to je kvadrat svakog realnog broja nenegativan. (Naravno, izraz na levoj strani
        je realan broj zato \v{s}to je $a \geq 0$ i $b \geq 0$.) Nakon kvadriranja dobijamo
        \[
            a - 2\sqrt{ab} + b \geq 0,
        \]
        odakle, nakon nekoliko elementarnih algebarskih transformacija, sledi
        \[
            \frac{a+b}{2} \geq \sqrt{ab},
        \]
        \v{s}to je i trebalo pokazati.
    \end{proof}

    Slede\'ce tvr\dj enje se dokazuje indukcijom, ali za njegov dokaz jo\v{s} uvek nemamo dovoljno
    \LaTeX-ovskih ve\v{s}tina, pa \'cemo ga izostaviti.

    \begin{theorem*}[\textbf{Nejednakost aritmeti\v{c}ke i geometrijske sredine}]
        Neka je $n \in \mathbb{N}, n \geq 2$, i neka su $a_1, a_2, \ldots, a_n \in \mathbb{R}$ takvi da je 
        $a_i \geq 0$ za sve $i \in \{1, 2, \ldots, n\}$. Tada je
        \[
            \frac{a_1 + a_2 + \ldots + a_n}{n} \geq \sqrt[n]{a_1 \cdot a_2 \cdot \ldots \cdot a_n}.
        \]
        Jednakost va\v{z}i ako i samo ako je $a_1 = a_2 = \ldots = a_n$.
    \end{theorem*}

    Interesantna je i slede\'ca:

    \begin{theorem*}[\textbf{Nejednakost geometrijske i harmonijske sredine}]
        Neka je $n \in \mathbb{N}, n \geq 2$, i neka su $a_1, a_2, \ldots, a_n \in \mathbb{R}$ takvi da je 
        $a_i \geq 0$ za sve $i \in \{1, 2, \ldots, n\}$. Tada je
        \[
            \sqrt[n]{a_1 \cdot a_2 \cdot \ldots \cdot a_n} 
            \geq 
            \frac{n}{\frac{1}{a_1}+\frac{1}{a_2}+\ldots+\frac{1}{a_n}}.
        \]
        Jednakost va\v{z}i ako i samo ako je $a_1 = a_2 = \ldots = a_n$.
    \end{theorem*}

    \section*{Zadatak 6b}

    Neka su $\vec{u} =(x_1, y_1)$ i $\vec{v} = (x_2, y_2)$ dva vektora u $\mathbb{R}^2$ odabrana tako da
    je $\vec{u} \neq \vec{0}$ i $\vec{v} \neq \vec{0}$. Du\v{z}ina vektora $\vec{u}$ data je sa
    \[
        |\vec{u}| = \sqrt{x_1^2 + y_1^2},
    \]
    dok je skalarni proizvod vektora $\vec{u}$ i $\vec{v}$ dat sa
    \[
        \vec{u} \cdot \vec{v} = x_1x_2 + y_1y_2.
    \]

    \begin{theorem*}
        Neka su $\vec{u} = (x_1, y_1)$ i $\vec{v} = (x_2, y_2)$ dva vektora u $\mathbb{R}^2$ odabrana
        tako da je $\vec{u} \neq \vec{0}$ i $\vec{v} \neq \vec{0}$. Tada je
        \[
            \cos \angle (\vec{u}, \vec{v}) = \frac{\vec{u} \cdot \vec{v}}{|\vec{u}| \cdot |\vec{v}|}.
        \]
    \end{theorem*}


    \section{Zadatak 6c}

    \begin{lemma*}
        Dokazati da je 
        \[
            a^k - 1 = (a - 1)(a^{k-1} + a^{k-2} + \ldots + a + 1)
        \]
        za svaki realan broj $a$ i svaki prirodan broj $k \geq 2$.
    \end{lemma*}
    \begin{proof}
        Direktnim ra\v{c}unom.
    \end{proof}

    \begin{theorem*}[\textbf{Mersenovi brojevi}]
        Ako je $2^n - 1$ prost broj onda je i $n$ prost broj, za svako $n \in \mathbb{N}$.
    \end{theorem*}
    \begin{proof}
        Pretpostavimo da $n$ nije prost broj. Ako je $n = 1$ onda je $2^n - 1 = 1$, \v{s}to nije 
        prost broj. Pretpostavimo, zato, da je $n \geq 2$. Kako $n$ nije prost broj, postoje prirodni
        brojevi $m \geq 2$ i $k \geq 2$ takvi da je $n = mk$. Sada je 
        \[
            2^n - 1 = 2^{mk} - 1 = (2^m)^k - 1 = a^k - 1,
        \]
        gde smo stavili $a = 2^m$ da bismo lak\v{s}e pratili nastavak dokaza. Prema Lemi je
        \[
            a^k - 1 = (a - 1)(a^{k-1} + a^{k-2} + \ldots + a + 1)
        \]
        i zato je 
        \[
            2^n - 1 = (a - 1)(a^{k-1} + a^{k-2} + \ldots + a + 1).
        \]
        Kako je $a = 2^m$ i $m \geq 2$ zaklju\v{c}ujemo da je $a-1 \geq 3$. S druge strane iz
        $k \geq 2$ i $a \geq 4$ sledi da je $a^{k-1} + a^{k-2} + \ldots + a + 1 \geq 5$. Dakle,
        $2^n - 1$ je slo\v{z}en broj.


    \end{proof}

\end{document}