\documentclass[10pt,a4paper]{article}

\usepackage[serbian]{babel}
\usepackage{amsthm}
\usepackage{amsmath}
\usepackage{amssymb}

\newcommand{\astop}{\mathop{\displaystyle\mathrm{\ast}}}
\newcommand{\impliesid}{\mathop{\Rightarrow}}

\newtheorem*{theorem*}{Teorema}

\begin{document}
% \[
%     f(x)\stackrel{\mathrm{def}}{=}\cos x
% \]

% \[
%     \sum_{
%     {I\subseteq \{1, 2 \ldots, n\}}
%     \atop{|I| \geq k}}a_I
% \]

% \[
%     \sum_{
%         {0 \leq i \leq r
%         \atop{0 \leq j \leq s}}
%         \atop{0 \leq k \leq t}}C(i, j, k)
% \]
% \[
%     \alpha \atopwithdelims<> \beta
% \]

% \[
%     \alpha \atopwithdelims() \beta
% \]

% \[
%     x \! x
% \]
% \[
%     x \; x
% \]
% \[
%     x \quad x
% \]
% \[
%     x \qquad x
% \]

% \[
%     \int_0^\infty e^{-x^2}\, dx=\sqrt{2\pi}
% \]
% % \[
% %     \int\!\!\int\limits_0^\infty e^{-x^2}\, dx=\sqrt{2\pi}
% % \]
% % $\int\limits_0^\infty e^{-x^2}\, dx=\sqrt{2\pi}$
% \[
%     \sum_{n=1}^\infty \frac{1}{n^2} = \frac{\pi^2}{6}
% \]
% \[
%     \int_a^b \, \textstyle {1 \overwithdelims() x} \, dx
% \]

% \[
%     \astop\limits_{i=1}^n f_i.
% \]

% Sizes

% \[
%     \Biggl( \biggl( \Bigl( \bigl( x + 1\bigr) \Bigr) \biggr) \Biggr)
% \]
% \[
%     y \Biggm/ x
% \]
\section{Zadatak 7a}

\textbf{Primer: Problem sa terijerima i kometama.} Iz slede\'ce tri premise izvesti
odgovaraju\'ci zaklju\v{c}ak:

\begin{enumerate}
    \item Nijedan terijer ne luta me\dj u zvezdama.
    \item Ni\v{s}ta \v{s}to ne luta me\dj u zvezdama nije kometa.
    \item Samo terijeri imaju kovrd\v{z}av rep.
\end{enumerate}

\noindent
\textit{Re\v{s}enje.} Uvedimo slede\'ce oznake:
\begin{align*}
    T &= \text{terijer}, &&L = \text{luta me\dj u zvezdama} \\
    K &= \text{kometa}, &&R = \text{ima kovrd\v{z}av rep}
\end{align*}

\noindent
Prethodne tri izjave mo\v{z}emo formalno da zapi\v{s}emo ovako:
\begin{enumerate}
    \item $T \Rightarrow \neg L$ (ako je terijer onda ne luta zvezdama);
    \item $\neg L \Rightarrow \neg K$ (ako ne luta me\dj u zvezdama onda nije kometa); i 
    \item $R \Rightarrow T$ (ako ima kovrd\v{z}av rep onda je to terijer).
\end{enumerate}

\noindent
Sada se lako uo\v{c}ava niz implikacija:
\[
    R \impliesid_{(3)} T \impliesid_{(1)} \neg L \impliesid_{(2)} \neg K
\]

\noindent
Dakle, $R \Rightarrow \neg K$. Imaju\'ci u vidu da je $R \Rightarrow \neg K \equiv K \Rightarrow \neg R$ zaklju\v{c}ujemo,
tako\dj e, da je $K \Rightarrow \neg R$, odnosno, da \emph{komete nemaju kovrd\v{z}av rep}.

\section{Zadatak 7b}
\textbf{Primer.} Gde je gre\v{s}ka u slede\'cem ``dokazu'' da je 1 = 2:
\begin{align*}
    -2 &= -2 \\
    1-3 &= 4-6 \\
    1+3 + \frac{9}{4} &= 4 - 6 + \frac{9}{4} \\
    \biggl( 1 - \frac{3}{2} \biggr)^2 &= \biggl(2 - \frac{3}{2} \biggr)^2 \\
    \sqrt{\biggl( 1 - \frac{3}{2} \biggr)^2} &= \sqrt{\biggl(2 - \frac{3}{2} \biggr)^2} \\
    1 - \frac{3}{2} &= 2 - \frac{3}{2} \\
    1 &= 2
\end{align*}

\pagebreak
\section{Zadatak 7c}
\begin{center}
    \large\textbf{Dokazi kontrapozicijom}
\end{center}

\vspace*{6px}

\normalsize
Nekada se de\v{s}ava da nije lako na\'ci direktan dokaz. Jedan na\v{c}in da doka\v{z}emo tvr\dj enje oblika
$p \Rightarrow q$ je da koristimo tautologiju
\[
    \models (p \Rightarrow q) \Leftrightarrow (\neg q \Rightarrow \neg p)
\]

\noindent
koja ka\v{z}e da je logi\v{c}ki svejedno da li \'cemo dokazivati da
\[
    p \Rightarrow q
\]

\noindent
ili da
\[
    \neg q \Rightarrow \neg p.
\]

\noindent
Dokazi zasnovani na ovoj strategiji se zovu \emph{dokazi kontrapozicijom}.

\vspace*{6px}
\noindent
\textbf{Primer.} \quad Neka je $n$ prirodan broj. Ako je $n^2$ paran onda je i $n$ paran broj.
\begin{proof}
    Treba da poka\v{z}emo slede\'ce:
    \[
        \underbrace{n^2 \; \text{je paran broj}}_p \Rightarrow \underbrace{n \; \text{je paran broj}}_q
    \]
    Kontrapozicija ovog tvr\dj enja glasi:
    \[
        \underbrace{n \; \text{je neparan broj}}_{\neg q} \Rightarrow \underbrace{n^2 \; \text{je neparan broj}}_{\neg p}
    \]
    i to se lako dokazuje. Neka je $n$ neparan broj. Tada je
    \[
        n = 2k + 1
    \]
    za neko $k \geq 0$. No, tada je
    \[
        n^2 = (2k + 1)^2 = 4k^2 + 4k + 1 = 2 \underbrace{(2k^2 + 2k)}_m + 1 = 2m + 1
    \]
    \v{s}to zna\v{c}i da je $n^2$ neparan broj.
\end{proof}

\section{Zadatak 7d}

\begin{theorem*}[Gvido Fubini]
    \normalfont
    Neka su $X$ i $Y$ prostori sa $\sigma$-kona\v{c}nom merom i neka je 
    $f: X \times Y \rightarrow \mathbb{R}$ merljiva funkcija takva da je 
    \[
        \int\limits_{X \times Y} |f(x, y)| d(x, y) < \infty
    \]
    Tada je
    \[
        \int\limits_X \Biggl(\int\limits_Y f(x, y)\; dy \Biggr) \; dx = 
        \int\limits_Y \Biggl(\int\limits_X f(x, y)\; dx \Biggr) \; dy =
        \int\limits_{X \times Y} f(x, y)\; d(x, y).
    \]
\end{theorem*}

\pagebreak

\begin{theorem*}[Srinivasa Ramanud\v{z}an]
    \normalfont
    Ako kompleksna funkcija $f$ ima razvoj u obliku
    \[
        f(x) = \sum_{k=0}^\infty \frac{\varphi (k)}{k!} (-x)^k
    \]
    za neko $\varphi$ onda je
    \[
        \int_0^\infty x^{s-1} f(x)\; dx = \Gamma (s) \varphi (-s),
    \]
    gde je $\Gamma(s)$ ozna\v{c}ena gama-funkcija.
\end{theorem*}

\end{document}