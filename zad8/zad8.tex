% Equation settings: leqno, fleqn
\documentclass[10pt, a4paper]{article}

\usepackage[serbian]{babel}

\usepackage[a4paper, total={5in, 8in}]{geometry}

\usepackage{amsmath}
\usepackage{amsthm}
\usepackage{amssymb}

\newtheorem{define}{Definicija}
\newtheorem{example}{Primer}

\begin{document}

% \[
%     \left[
%     \begin{array}[]{lcr}
%         x-\lambda & 1 & 0 \\
%         0 & x - \lambda & 1 \\
%         0 & 0 & x - \lambda
%     \end{array}
%     \right]
% \]

% \begin{equation}
%     a^2 = b^2 + c^2
% \end{equation}
% \[
%     \frac{\Gamma \vdash B}{\Gamma, !A \vdash B}
%     \eqno\mathrm{(weakening)}    
% \]

% \begin{eqnarray}
%     E &=& \hbar \cdot \nu \\
%     E &=& m \cdot c^2
% \end{eqnarray}

\section{Determinante}

\begin{define}
    Neka je n prirodan broj i neka su $a_ij$ proizvoljni realni brojevi, $1 \leq i, j \leq n$. 
    Determinanta reda $n$ je algebarski izraz oblika
    \[
        D = \left|
            \begin{array}[]{lccr}
            a_{11} & a_{12} & \ldots & a_{1n} \\
            a_{21} & a_{22} & \ldots & a_{2n} \\
            \vdots & \vdots & & \vdots \\
            a_{n1} & a_{n2} & \ldots & a_{nn}
            \end{array}
            \right| = 
            \sum_{f \in S_n} (-1)^{\mathrm{inv}(f)} a_{1f_1} a_{2f_2} \ldots a_{nf_n},
    \]
    gde je $S_n$ skup svih permutacija skupa $\{1, 2, \ldots, n\}$. Kra\'ce pi\v{s}emo i $D = |a_{ij}|_{n \times n}$.
\end{define}

\section{Matrice}
\begin{define}
    {\normalfont Proizvod} kompatibilnih matrica $A = [a_{ij}]_{k \times n}$ i $B = [b_{ij}]_{n \times m}$,
    pi\v{s}emo $C = AB$, je matrica $C = [c_{ij}]_{k \times m}$ \v{c}iji se elementi ra\v{c}unaju na slede\'ci na\v{c}in:
    \[
        C_{ij} = \sum_{s=1}^n a_{is}b_{sj}.
    \]
    Ova formula prakti\v{c}no zna\v{c}i da se element na mestu $(i, j)$ matrice $C$ dobija tako \v{s}to se
    $i$-ta vrsta matrice $A$ {\normalfont skalarno pomno\v{z}i} $j$-tom kolonom matrice B:
    \[
        \left[
        \begin{array}[]{ccccc}
            a_{11} & a_{12} & a_{13} & \ldots & a_{1n} \\
            \vdots & \vdots & \vdots & & \vdots \\
            \hline
            a_{i1} & a_{i2} & a_{i3} & \ldots & a_{in} \\
            \hline
            \vdots & \vdots & \vdots & & \vdots \\
            a_{k1} & a_{k2} & a_{k3} & \ldots & a_{kn} \\
        \end{array}
        \right]
        \cdot
        \left[
        \begin{array}[]{cc|c|cc}
            b_{11} & \ldots & b_{1j} & \ldots & b_{1m} \\
            b_{21} & \ldots & b_{2j} & \ldots & b_{2m} \\
            b_{31} & \ldots & b_{3j} & \ldots & b_{3m} \\
            \vdots & \vdots & \vdots & & \vdots \\
            b_{k1} & \ldots & b_{nj} & \ldots & b_{nm} \\
        \end{array}
        \right] =
        \left[
        \begin{array}[]{ccccc}
            c_{11} & \ldots & c_{1j} & \ldots & c_{1m} \\
            \vdots & \vdots & \vdots & & \vdots \\
            c_{i1} & \ldots & \overline{\underline{|c_{ij}|}} & \ldots & c_{3m} \\
            \vdots & \vdots & \vdots & & \vdots \\
            c_{k1} & \ldots & c_{kj} & \ldots & c_{nm} \\
        \end{array}
        \right].
    \]  

\end{define}

\pagebreak
\section{Inverzna matrica}
\begin{example}
    {Izra\v{c}unati inverznu matricu matrice} $A = \left[\begin{array}[]{cccc}
        1 & 1 & 2 & 2 \\
        2 & 3 & 3 & 5 \\
        2 & 1 & 4 & 3 \\
        4 & 3 & 7 & 6 \\
    \end{array}\right]$.
\end{example}
\begin{proof}[Re\v{s}enje]
    Da bismo izra\v{c}unali inverznu matricu matrice $A$ dovoljno je re\v{s}iti po $X$ matri\v{c}nu
    jedna\v{c}inu $AX = E$, jer \'ce tada biti $X = A^{-1}$. Dakle, treba re\v{s}iti slede\'cu matri\v{c}nu jedna\v{c}inu:
\[
    \left[ 
    \begin{array}[]{cccc}
        1 & 1 & 2 & 2 \\
        2 & 3 & 3 & 5 \\
        2 & 1 & 4 & 3 \\
        4 & 3 & 7 & 6 \\
    \end{array}
    \right] \cdot
    \left[ 
    \begin{array}[]{cccc}
        x_{11} & x_{12} & x_{13} & x_{14} \\
        x_{21} & x_{22} & x_{23} & x_{24} \\
        x_{31} & x_{32} & x_{33} & x_{34} \\
        x_{41} & x_{42} & x_{43} & x_{44} \\
    \end{array}
    \right] = 
    \left[ 
    \begin{array}[]{cccc}
        1 & 0 & 0 & 0 \\
        0 & 1 & 0 & 0 \\
        0 & 0 & 1 & 0 \\
        0 & 0 & 0 & 1 \\
    \end{array}
    \right]
\]
\end{proof}

Ako strpljivo raspi\v{s}emo proizvod matrica na levoj strani, dobijamo da treba simultano re\v{s}iti
\v{c}etiri sistema jedna\v{c}ina koji svi imaju istu matricu sistema:

\begin{equation*}
    \begin{aligned}[l]
        x_{11} + x_{21} + 2x_{31} + 2x_{41} &=& 1 \\
        2x_{11} + 3x_{21} + 3x_{31} + 5x_{41} &=& 0 \\
        2x_{11} + x_{21} + 4x_{31} + 3x_{41} &=& 0 \\
        4x_{11} + 3x_{21} + 7x_{31} + 6x_{41} &=& 0
    \end{aligned}
    \qquad
    \begin{aligned}[r]
        x_{12} + x_{22} + 2x_{32} + 2x_{42} &=& 0 \\
        2x_{12} + 3x_{22} + 3x_{32} + 5x_{42} &=& 1 \\
        2x_{12} + x_{22} + 4x_{32} + 3x_{42} &=& 0 \\
        4x_{12} + 3x_{22} + 7x_{32} + 6x_{42} &=& 0
    \end{aligned}
\end{equation*}
\\
\begin{equation*}
    \begin{aligned}[l]
        x_{13} + x_{23} + 2x_{33} + 2x_{43} &=& 0 \\
        2x_{13} + 3x_{23} + 3x_{33} + 5x_{43} &=& 0 \\
        2x_{13} + x_{23} + 4x_{33} + 3x_{43} &=& 1 \\
        4x_{13} + 3x_{23} + 7x_{33} + 6x_{43} &=& 0
    \end{aligned}
    \qquad
    \begin{aligned}[r]
        x_{14} + x_{24} + 2x_{34} + 2x_{44} &=& 0 \\
        2x_{14} + 3x_{24} + 3x_{34} + 5x_{44} &=& 0 \\
        2x_{14} + x_{24} + 4x_{34} + 3x_{44} &=& 0 \\
        4x_{14} + 3x_{24} + 7x_{34} + 6x_{44} &=& 1
    \end{aligned}
\end{equation*}

Primeni\'cemo Gaus-\v{Z}ordanovu eliminaciju:
\[
    \left[
        \begin{array}[]{cccc|cccc}
            1 & 1 & 2 & 2 & 1 & 0 & 0 & 0 \\
            2 & 3 & 3 & 5 & 0 & 1 & 0 & 0 \\
            2 & 1 & 4 & 3 & 0 & 0 & 1 & 0 \\
            4 & 3 & 7 & 6 & 0 & 0 & 0 & 1 \\
        \end{array}
    \right]
    \sim \ldots \sim
    \left[
        \begin{array}[]{cccc|rrrr}
            1 & 0 & 0 & 0 & -7 & 1 & 1 & 1 \\
            0 & 1 & 0 & 0 & 4 & -1 & -3 & 1\\
            0 & 0 & 1 & 0 & 4 & -1 & -1 & 0 \\
            0 & 0 & 0 & 1 & -2 & 1 & 2 & -1\\
        \end{array}
    \right]
\]

Prema tome, $A^{-1} = \left[ \begin{array}{rrrr}
    -7 & 1 & 1 & 1 \\
    4 & -1 & -3 & 1\\
    4 & -1 & -1 & 0 \\
    -2 & 1 & 2 & -1\\
 \end{array}\right]$

\end{document}